\section*{Abstract}
This paper investigates the relationship between springs and the covalent bonds that form between atoms in molecules. Moreover the formation of covalent bonds in molecules is explained by looking at atomic and molecular orbitals. Furthermore the paper sheds light on infrared-spectroscopy as a tool to identification of chemical compounds. The paper includes a derivation of Hooke's law and the position function x(t] and the frequency of a harmonic oscillator. Four different experiments are made to conclude the following; 1. the validity of Hooke's law. 2. that a  mass attached to a spring performs a harmonic oscillating movement when displaced from its state of equilibrium. 3. that a system of 2 springs' stiffnesses, k, is equal to the sum of each spring's stiffness. 4. That a system of 3 springs stiffnesses is equal to the sum of each of the springs stiffnesses. 
In the end of the paper the method of using infrared-spectroscopy as an identification tool is taken into use and three different chemical compounds, respectively - Paracetamol, Ibuprofen and R-limonen is paired to three different infrared spectra. 
