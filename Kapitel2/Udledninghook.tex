\subsection{Udledning af Hookes lov}\label{Hookes}
Betragt den potentielle energi af en fjeder givet ved

\bigskip

\begin{equation}
V=\dfrac{1}{2} \cdot k \cdot (r-r_{0})^2
\end{equation}

\bigskip

Sæt nu $r_0 = 0$ og lad den initiale længde $r_i = r$ og den endelige udstrækningslængde $r_e = r + \Delta r$. Hvis vi ser på ændringen i den potentielle energi for systemet får vi da:

\bigskip

\begin{equation}
\Delta V = V_e - V_i = \dfrac{1}{2} \cdot k \cdot (r+ \Delta r)^2 - \dfrac{1}{2} \cdot k \cdot r^2
\end{equation}

\bigskip

Vi skriver $(r + \Delta r)^2$ ud og får:

\bigskip

\begin{equation}
\Delta V = V_e - V_i = \dfrac{1}{2} \cdot k \cdot (r^2 + 2r \Delta r + \Delta r^2) - \dfrac{1}{2} \cdot k \cdot r^2 
\end{equation}

\bigskip

I ligningen betragter vi nu de to størrelser $\dfrac{1}{2} \dot k \cdot r^2 og - \dfrac{1}{2} \cdot k \cdot r^2$. Disse to størrelser går ud med hinanden og efterlader os med

\bigskip

\begin{equation}
\Delta V = kr \Delta r + \dfrac{1}{2} \cdot k \cdot \Delta r^2
\end{equation}

Vi noterer os nu, at det første led på højresiden af lighedstegnet er lineært i $\Delta r$ og at det andet led på højresiden af lighedstegnet er kvadreret i $\Delta r$. $\Delta r$ er en infinitesimal i kalkulus, hvilket betyder at størrelsen $\Delta r$ er meget mindre end 1. Når man så vælger at kvadrere noget der er strengt mindre end 1 og $\Delta r$ er så lille som muligt, hvilket er naturligt i kalkulusregning, går størrelsen $\dfrac{1}{2} \cdot k \cdot \Delta r^2$ mod 0. Så vi ender med

\bigskip

\begin{equation}
\Delta V = kr \Delta r
\end{equation} 

\bigskip

Dette udtryk beskriver altså ændringen i den potentielle energi når vi udstrækker fjederen $\Delta r$, der er en meget lille størrelse. 

\bigskip

Vi ser nu på ændringen for energi i et system, der er givet ved formlen 
\begin{equation}
\Delta E = \Delta E_kin + \Delta E_pot
\end{equation}

\bigskip

Ændringen i energi for et system er lig summen af ændringen i potentiel energi og kinetisk energi. 

Dette skulle meget gerne være lig 0 og vi får da 

\bigskip

\begin{equation}
\Delta E = \Delta E_{kin} + \Delta V = 0
\end{equation}

\bigskip

Hvilket, hvis størrelsen for den potentielle energi givet ved $\Delta V = kr \Delta r$ substitueres ind i stedet for $V$ i ligning  giver

\bigskip

\begin{equation}
\Delta E_{kin} + kr \Delta r = 0 \rightarrow \Delta E_kin = -kr \Delta r
\end{equation}

\bigskip

Nu vil vi bruge det fysiske resultat

\bigskip

\begin{equation}
\Delta E_{kin} = A = \textbf{F} \times \Delta \textbf{r} = F \cdot \Delta r (Arbejde-Energi teoremet)
\end{equation}

\bigskip

Notér at der i dette tilfælde regnes på en fjeder og der da kan ses bort fra at $F$ og $\Delta r$ begge normalt er vektorer og arbejdet da skal beregnes som krydsproduktet af de to vektorer da fjederen udstrækkes i samme retning som fjederen trækkes. Vi tillader os da at regne arbejdet som $F \cdot \Delta r = A$.

Nu bruger vi resultatet 

\bigskip

\begin{equation}
\Delta E_{pot} = \textbf{F} \times \Delta \textbf{r} = F \cdot \Delta \textbf{r} = -kr \Delta r
\end{equation}

\bigskip

Ved division på begge sider af lighedstegnet af ligningen 

\bigskip

\begin{equation}
F \cdot \Delta r = -kr \Delta r
\end{equation}

\bigskip

Får vi det ønskede resultat: 

\bigskip

\begin{equation}
F = -k \cdot r
\end{equation}