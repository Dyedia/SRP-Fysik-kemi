\subsection{Udledning af Hookes lov}\label{Hookes}
Betragt den potentielle energi af en fjeder givet ved

\bigskip

\begin{equation}
V=\dfrac{1}{2} \cdot k \cdot (x-x_{0})^2
\end{equation}

\bigskip

Sæt nu $x_0 = 0$ og lad den initiale længde $x_i = x$ og den endelige udstrækningslængde $x_e = x + \Delta x$. Hvis vi ser på ændringen i den potentielle energi for systemet får vi da:

\bigskip

\begin{equation}
\Delta V = V_e - V_i = \dfrac{1}{2} \cdot k \cdot (x+ \Delta x)^2 - \dfrac{1}{2} \cdot k \cdot x^2
\end{equation}

\bigskip

Vi skriver $(x + \Delta x)^2$ ud og får:

\bigskip

\begin{equation}
\Delta V = V_e - V_i = \dfrac{1}{2} \cdot k \cdot (x^2 + 2x \Delta x + \Delta x^2) - \dfrac{1}{2} \cdot k \cdot x^2 
\end{equation}

\bigskip

I ligningen betragter vi nu de to størrelser $\dfrac{1}{2} \dot k \cdot x^2 og - \dfrac{1}{2} \cdot k \cdot x^2$. Disse to størrelser går ud med hinanden og efterlader os med

\bigskip

\begin{equation}
\Delta V = kx \Delta x + \dfrac{1}{2} \cdot k \cdot \Delta x^2
\end{equation}

Vi noterer os nu, at det første led på højresiden af lighedstegnet er lineært i $\Delta x$ og at det andet led på højresiden af lighedstegnet er kvadreret i $\Delta x$. $\Delta x$ er det vi betegner som en infinitesimal i kalkulus, hvilket betyder at størrelsen $\Delta x$ uendeligt lille. Når man så vælger at kvadrere noget der er så småt som $\Delta x$ vil størrelsen $\dfrac{1}{2} \cdot k \cdot \Delta x^2$ gå mod 0. Så vi ender med

\bigskip

\begin{equation}
\Delta V = kx \Delta x
\end{equation} 

\bigskip

Dette udtryk beskriver altså ændringen i den potentielle energi når vi udstrækker fjederen $\Delta x$, der er en uendelig lille udstrækning. 

\bigskip

Vi ser nu på ændringen af energi i et system, der er givet ved formlen 
\begin{equation}
\Delta E = \Delta E_{kin} + \Delta E_{pot}
\end{equation}

\bigskip

Ændringen i energi for et system er lig summen af ændringen i potentiel energi og kinetisk energi. 

Da dette er 0 for vores isolerede system får vi: 

\bigskip

\begin{equation}
\Delta E = \Delta E_{kin} + \Delta V = 0
\end{equation}

\bigskip

Hvilket, hvis størrelsen for den potentielle energi givet ved $\Delta V = kx \Delta x$ substitueres ind i stedet for $V$ i ligning  giver

\bigskip

\begin{equation}
\Delta E_{kin} + kx \Delta x = 0 \rightarrow \Delta E_{kin} = -kx \Delta x
\end{equation}

\bigskip

Nu vil vi bruge det fysiske resultat

\bigskip

\begin{equation}
\Delta E_{kin} = A = \textbf{F} \times \Delta \textbf{r} = F \cdot \Delta x (Arbejde-Energi teoremet)
\end{equation}

\bigskip

Notér at der i dette tilfælde regnes på en fjeder og der da kan ses bort fra at $F$ og $\Delta x$ vektorer. Dette kan vi tillade os da kraften virker i samme retning som fjederen trækkes. Vi tillader os da at regne arbejdet som $F \cdot \Delta x = A$.

Nu bruger vi resultatet 

\bigskip

\begin{equation}
\Delta E_{pot} = \textbf{F} \times \Delta \textbf{x} = F \cdot \Delta \textbf{x} = -kx \Delta x
\end{equation}

\bigskip

Ved division på begge sider af lighedstegnet af ligningen 

\bigskip

\begin{equation}
F \cdot \Delta x = -kx \Delta x
\end{equation}

\bigskip

Får vi det ønskede resultat: 

\bigskip

\begin{equation}
F = -k \cdot x
\end{equation}