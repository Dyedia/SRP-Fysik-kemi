\section{Konklusion}
Med udgangspunkt i de forsøg som blev foretaget og behandlet i denne opgave står det nu klart, at Hookes lov gør sig gældende for \emph{perfekte} fjedre. Derudover har det også vist sig at positionen for objekter bundet til fjedre, der sættes i bevægelse foretager harmonisk oscillerende bevægelser, som kan beskrives ud fra en funktion på formen: $x(t)=A \cdot (\omega t + C) + D$. Af dette resultat kunne frekvensen for en harmonisk oscillerende bevægelse bestemmes som: $f=\frac{1}{2\pi} \cdot \sqrt[2]{\frac{k}{m}}$, der i afsnittet om bestemmelse af bølgetal relaterede sig direkte til bestemmelsen af bølgetallet \={v}. Teorien om de karakteristiske egenfrekvenser for funktionelle grupper gjorde det muligt gennem en analyse af de funktionelle grupper i paracetamol, ibuprofen og R-limonen at kunne knytte de tre stoffer til hvert deres IR-spekrum.