\subsection{Harmoniske svingninger}\label{teori: Harmoniske svingninger}
En oscillerende bevægelse er en bevægelse hvor et objekt vil bevæge sig frem og tilbage over sin hviletilstand. Dette kan eksempelvis ses hos penduler og masser bundet til fjedre.
\\

\begin{wrapfigure}{r}{0.3\textwidth}
\centering
\includegraphics[scale=1]{Billeder/fjeder}
\caption{Fjeder med lod \label{fig:fjeder}}
\end{wrapfigure} 

Eksemplet der vil blive betragtet i dette kapitel er oscillerende bevægelse, der er hensigtsmæssig i forhold til at forstå IR-spektroskopi. Vi vil se på en masse der er bundet til en fjeder. se figur \ref{fig:fjeder}.

Med udgangspunkt i figur \ref{fig:fjeder} vil vi se på en masse der i den ene ende er fastgjort til noget stationært og i den anden ende er fastgjort til en kugle med en masse. Når kuglen ikke påvirkes af andre ydre kræfter end tyngdekraften vil kuglen opnå en hviletilstand hvor fjederkraften er lige så stor som tyngdekraften. Vi sætter denne hviletilstand til $x_0=0$. Når vi så begynder at påvirke system med en ydre kraft ved f.eks. at hive i kuglen vil fjederen svare tilbage ved at trække endnu hårdere i kuglen. Hvis vi betrager opad som den positive retning vil vi på figur \ref{fig:fjeder} have flyttet kuglen en afstand $-x$ væk fra sin hvileposition og fjederkraften givet ved Hookes lov, der blev beskrevet i \ref{Hookes} vil være lig $F_x=-k \cdot -x$. Hvor F er fjederkraften, k er fjederkonstanten for den givne fjeder og x er afstanden flyttet fra hvilepositionen $x_0$.