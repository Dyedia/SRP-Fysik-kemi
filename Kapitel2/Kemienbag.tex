\section{Den bagvedlæggende kemi}

Her har du brug for harmoniske svingninger \ref{teori: Harmoniske svingninger}

Stort set alle kemiske forbindelser, der indeholder kovalente bindinger absorberer nogle specielle frekvenser af elektromagnetisk stråling i den infrarøde del af det elektromagnetiske spektrum. Det infrarøde lys har en længere bølgelængde en det lys menneskets øjne kan opfange $400nm - 800nm$ men kortere end det vi bruger i vores mikroovne, der er længere end 1mm. Det interval vi beskæftiger os med når vi ser på IR-spektroskopi ligger i intervallet $2,5 \mu m- 25 \mu m $. Det er de bølgelængder af det infrarøde lys, som giver den vibrationelle effekt på atomerne og molekylerne som vi søger. Når vi i afsnit (MANGLER) skal se på de 3 forskellige IR-spektra, vil der hen af x-aksen være plottet reciprokke centimeter i stedet. Dette skyldes, at kemikere af konvention laver bølgelængder om til \textbf{bølgetal}, \={v} , ved at tage den reciprokke værdi af bølgelængden i cm. 

\begin{center}
\={v}$= \dfrac{1}{\lambda}$
\end{center}

Den primære grund til at vi vælger at bruge bølgetal i stedet for en bølgelængde er, at bølgetallene er direkte proportionelle til energien af de fotoner, som lyset består af. På den måde vil et højere bølgetal repræsenterer en højere energi. Bølgetal kan laves om til en frekvens ved at gange det med lysets hastighed, c, for så at puttes ind i formlen for fotonenergi $E_{fot} = h \cdot \upsilon$

Dette ses da at fotonenergien bliver større ved et større bølgetal 
\\

\begin{center}
$\upsilon =$ \={v} $\cdot c \rightarrow$ \={v} $\cdot c \cdot h = E_{fot}$
\end{center}

Som med alle former for stråling bliver molekyler ramt af stråling exciteret til et højere energiniveau ved absorbtion af den infrarøde stråling. Det resultat der gør IR-spektroskopi interessant er, at molekyler absorberer energi af forskellige frekvenser. De frekvenser som molekylerne absorberer svarer lige præcist til de egenfrekvenser som molekylerne får når atomerne, der er bundet sammen af  kovalente bindinger, enten bøjer eller svinger som vist på figur (MANGLER FIGUR DER ILLUSTRERER SVING OG BØJ). Dette er med forbehold for, at der er visse molekyler, der ikke giver sig til kende på IR-spektra grundet årsager som vi skal behandle i et senere afsnit. 
\\

Siden alle slags bånd har en forskellig egenfrekvens, i og med at de befinder sig i lidt forskellige $miljøer$, vil to molekyler give forskellige udslag på et infrarødt spektrum. Selvom nogle absorbtionsmønstre kan være de samme, hvis begge molekyler eksempelvis indeholder en fælles funktionel gruppe, vil de to molekyler stadigvæk give nogle udslag som adskiller sig fra hinanden. IR-spektra kan på den måde betragtes den kemiske pendant til menneskets fingeraftryk.  \\

Et andet og i virkeligheden mere interessant resultat af IR-spektroskopi er, at vi kan opnå strukturel information om det pågældende molekyle vi laver spektroskopien på. 




