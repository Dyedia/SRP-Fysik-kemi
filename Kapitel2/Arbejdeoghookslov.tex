\chapter{IR-Spektroskopi}

\section{Fysikken bag}
\subsection{Arbejde og Hookes lov}
I den daglige omtale er arbejde et fysisk aktivitet der kræver energi. I fysikken bruger vi derimod arbejde som et udtryk for en kraft der virker på en genstand over en given distance.\refUniphys{146} Arbejdet, A, som en genstand er blevet tilført på en strækning, x, med en kraft, F, er givet ved formlen
\\
\begin{center}
 $A=F \cdot x$. 
\end{center}
\bigskip

Dette er dog under antagelse af at kraften der virker på objektet er konstant samt at flytningen af objektet er i samme retning som kraften virker. 
Vi støder dog ind i problemer med at bestemme energien ud fra denne formel, hvis vi ser på et system hvor kraften bliver større jo længere væk fra startstedet, $x_0$, objektet bevæger sig. Formod nu, at vi ser på et objekt der bevæger sig langs en ret linje og hvor kraften bliver større jo længere vi flytter objektet. Da vil det endelige samlede arbejde, $\sum A$ være lig summmen af alle de kræfter der har virket på objektet da størrelsen \textbf{arbejde} er en skalar.
\\
\begin{center}
$\sum A = A_1 + A_2 + ... + A_{n-1} + A_n$
\end{center}
\bigskip

For at bestemme denne størrelse deler vi den afstand, $x$, som objektet bevæger sig op i mindre dele $\Delta x_1$, $\Delta x_2$ osv. Vi deler disse afstande op i tilstrækkeligt små længder sådan at kraften der virker over de enkelte afstande er omtrent konstant. På denne måde kan vi beregne det totale arbejde som:
\\
\begin{center}
$A = F_1 \cdot \Delta x_1 + F_2 \cdot \Delta x_2 + .. + \cdot F_n \cdot \Delta  x_n$
\end{center}
\bigskip

Når vi får tilstrækkeligt mange længder og længderne samtidigt går mod at blive uendeligt små vil denne størrelse gå mod at blive integralet af F fra $x_1$ til $x_n$: 

\begin{center}
$A = \int\limits_{x_1}^{x_n} F dx$ 
\end{center}
\bigskip

Det er netop dette princip som Hookes lov benytter sig af. Hookes lov siger, at for en perfekt fjeder vil kraften være direkte proportional til afstanden som fjederen er udtrukket. 

\begin{center}
$F = -kx$
\end{center}
\bigskip

hvor k er stivheden er fjederen (eller fjederkonstanten), F er kraften og x er den længden som fjederen er trukket ud. Med en perfekt fjeder skal der forstås at denne formel gælder uanset hvor langt man strækker fjederen ud eller presser den sammen. Dette er naturligvis en grov antagelse, men det er en god fysisk beskrivelse for at beregne størrelsen af en fjederkraft. Vi vil nu se på udledningen af denne sammenhæng.

Der vil her blive taget udgangspunkt i formlen 
\begin{equation}
V=\dfrac{1}{2} \cdot k \cdot (x-x_{0})^2
\end{equation}

Hvor V beskriver den potentielle energi en fjeder har, når den er strukket en distance x ud fra hvilelængden $x_0$. 
