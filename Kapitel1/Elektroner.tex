\section{Elektroner}\label{elektroner}
Fra fysikkens verden ved vi følgende ting om elektroner. 

\begin{enumerate}
\item En elektrons position kan ikke bestemmes nøjagtigt. Det vi til gengæld kan sige noget om, er den orbital som elektronen befinder sig i. Vi kan bestemme størrelsen og formen på den orbital hvor der er en relativt stor chance for at finde den enkelte elektron i.

\item De orbitaler som elektronerne bevæger sig i karakteriseres ved kvantenumre $n = 1,2,3 ...$. Når n bliver større bliver afstanden til kernen større. Tilsvarende bliver energierne for orbitalerne også større med et større n. For n større end 1 kan der godt være forskellige orbitaler, som har samme værdi for n. En elektronskal indeholder orbitaler med samme værdi for n. Vi kan derfor vælge at benævne skallerne ved deres kvantenumre $1,2,3,4,5,6,7$ eller respektivt med bogstaverne $K,L,M,N,O,P,Q$. 

\item Orbitaler der har den samme værdi for n har nødvendigvis forskellige former. Se $2s$ og $2p$ orbitalerne, der behandles i afsnit \ref{sec:sp}

\item Da elektroner er fermioner har de en spin-værdi på enten $\frac{1}{2}$ eller $-\frac{1}{2}$ som respektivt er værdierne for spin-op og spin-ned elektroner, der tit og ofte skrives som små pile der vender opad eller nedad.

\item Der kan maksimalt være 2 elektroner i hver orbital og de har hver deres spin.
\end{enumerate}

Det er desuden værd at notere sig, at der for atomer med mere end 2 elektroner gælder, at energien for elektronerne i orbitalerne at:

\bigskip
\begin{center}
$s$ elektroner $<$ $p$ elektroner $<$ $d$ elektroner $<$ $f$ elektroner
\end{center}
\bigskip

Da der kan være 2 elektroner i første skal, 8 i anden, 18 i tredje, 32 i den fjerde osv.. og atomer som udgangspunkt gerne vil have en så lav energi som muligt vil elektronerne starte med at fylde de orbitaler ud med lavest energiniveau først. Måden hvorpå elektronerne fylder orbitalerne ud er ved først at fylde en spin-op eller spin-ned elektron i orbitalen og dernæst det modsatte i orbitalen. Dette gør sig gældende for s-orbitalerne, men for de andre orbitaler, p, d og f orbitalerne, vil der først blive puttet en elektron i hver orbital før de fortsættes med at blive fyldt op. Dette leder os til orbitaler.