\chapter{Kovalente bindinger og organiske molekyler}

Mange kemiske forbindelser indeholder ikke ioner, men består i stedet af atomer, som er bundet tæt sammen i det vi kalder molekyler. Disse bånd, der holder atomerne sammen, formes når de omkredsende elektroner deles af de to kerner og derfor skaber noget stabilitet. Et sådan bånd kaldes et kovalent bånd. I modsætning til ion-bindinger dannes kovalente bindinger når ingen af de to atomkerner der indgår i bindingen har en stor nok elektronegativitet til at $rive$ elektronen væk fra det andet atom. Et af de simpleste eksempler på en kovalent binding er molekylet $H_2$. Kovalente bindinger er af stor betydning for den infrarøde-spektroskopi på grund af kovalente bindinger kan betragtes som fjedre som vi fra fysikkens verden kender meget til.

Vi vil se nærmere på dette i de følgende afsnit men først vil vi indledningsvist snakke om atomernes elektronkonfiguration. 

