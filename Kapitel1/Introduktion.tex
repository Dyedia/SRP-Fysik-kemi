\chapter*{Indledning}
Infrarød-spektroskopi er et vigtigt værktøj indenfor kemien, da det i samfundet kan bruges til en bred vifte af ting. Her kan nævnes kvalitetskontrol af kemiske produkter, bestemmelse af alkoholprocent i blodet og til sikring mod gaslækager. 
\\

Denne opgave vil beskæftige sig med infrarød-spektroskopi som en metode brugt af kemikere til at identificere kemiske forbindelser. Den infrarøde-spektroskopi udnytter at bindinger mellem atomer i molekyler har en egenfrekvens som de vibrerer med. Disse vibrerende systemer kan i visse tilfælde betragtes som fjedre, som vi fra fysikkens side kender rigtigt godt til. Baggrunden for dannelsen af kovalente bindinger beskrives og ud fra Hookes lov, der i opgaven undersøges - både teoretisk og praktisk - sættes en formel, som skal kunne bestemme bølgetal, \={v}, for bestemte funktionelle grupper.


\chapter{Kovalente bindinger}
Mange kemiske forbindelser indeholder ikke ioner, men består i stedet af atomer, som er bundet tæt sammen i det vi kalder molekyler. De bånd, der holder atomerne sammen, formes når de omkredsende elektroner deles af de to kerner og skaber stabilitet. Et sådan bånd kaldes et kovalent bånd. I modsætning til ion-bindinger dannes kovalente bindinger når ingen af de to atomkerner, der indgår i bindingen har en stor nok elektronegativitet til at $rive$ elektronen væk fra det andet atom. Et af de simpleste eksempler på kovalent binding er i molekylet $H_2$. Kovalente bindinger er af stor betydning for den infrarøde-spektroskopi på grund af kovalente bindinger kan betragtes som fjedre.
\\
Vi vil se nærmere på dette i de følgende afsnit men vi vil indledningsvist lige snakke om atomernes elektronkonfiguration. 

