\section{MO-teori}
I molekylær orbital teori, forkortet MO-teori, ses der på fordelingen af elektroner i molekyler på samme måde som vi ser på molekyler i atomer (FODNOTE TIL GENERAL CHEMISTRY s.149 mangler). Måden hvorpå vi bestemmer hvordan en elektron opfører sig på i et molekyle er ved hjælp af kvantemekanik og en bølgefunktion $\Psi$. På denne måde kan energien af en elektron og formen på det område hvori en elektron bevæger sig i bestemmes. Dette vil jeg kun berøre sporadisk. Som vi i afsnit 1.1 fandt ud af at elektroner i forskellige energiniveauer havde størst chance for at eksistere i forskellige områder som vi kaldte orbitaler, ligeså har elektroner i molekyler også orbitaler som de har en stor chance for at eksistere i. Forskellen er bare, at i molekyler kan elektronerne findes nær kernen på en hvilken som helst kerne, der indgår i molekylet; hvilket er hvorfor vi kalder disse orbitaler for molekylære elektronorbitaler eller bare molekylære orbitaler. 
Ligesom atomare orbitaler er molekyære orbitaler også fyldt når de indeholder præcist 2 orbitaler med modsat spin. 
