\section{Orbitaler}
For at opnå en dybere forståelse af hvorfor atomer vælger at gå sammen om at danne kovalente bindinger ser vi os nødsaget til lige at betragte et enkelt atoms elektronkonfiguration. De subatomare entiteter som elektroner er opfører sig ikke lige så pænt som Bohr formulerede i hans atommodel, hvor elektronerne bevæger sig i cirkulære baner, eller skaller, rundt omkring atomets kerne. Ifølge nyere teori har elektronerne en chance for at eksistere i visse områder omkring atomet som forklaret i forrige afsnit. Det laveste energiniveau en elektron kan befinde sig i, når det bevæger sig rundt om atomkernen er kugleformet og har tildelt navnet 1s orbitalen. Figur \ref{fig:1sorbital} er en skitse af 1s orbitalen plottet i det kartesiske plan. 

\begin{figure}[ht!]
  \centering
  \begin{tikzpicture}[scale=2]
  \begin{scope}[font=\itshape]% to not type it every time, but better go for math mode


  \draw[-latex] (5,0)--(7,0) node[above]{x};
  \draw[-latex] (5,0)--(4.5,-0.7) node[above]{y};
  \draw[-latex] (5,0)--(5,1) node[above]{z};
  \orbital[color = purple, pos = {(5,0)}]{s} 
  \node[above] at (5.7,0.7) {1s orbital};

 		
  \end{scope}

  % correctly setting the background layer
  \begin{pgfonlayer}{backbackground}
  \fill[white](current bounding box.south west)rectangle
  (current bounding box.north east);
  \end{pgfonlayer}
\end{tikzpicture}
\caption{Abe \label{fig:1sorbital}}
\end{figure}

I s-orbitalerne er der plads til netop 2 elektroner. At den første orbital hedder 1s henviser til at den tilhører skal 1. I 2. skal tilhører elektronerne 2s orbitalen. S på grund af orbitalerne som elektronerne forventes at være i har den samme form som 1s orbitalen og 2 fordi orbitalen knytter sig til den 2. skal. Orbitalen ligner 1s orbitalen til forveksling, men vil have en større radius da elektronerne der befinder sig i denne orbital har mere energi. Da vi ved, at der skal være 8 elektroner i den 2. skal for, at oktetreglen er opfyldt og atomerne derfor er stabile mangler vi stadigvæk at placere 6 elektroner. Disse elektroner vil befinde sig i de såkaldte p-orbitaler. Der eksisterer 3 forskellige p-orbitaler, som ligner sløjfer og som hver især indeholder 2 elektroner. De tre p-orbitaler benævnes respektivt $p_x$, $p_y$ og $p_z$, de er ortogonale på hinanden og ligger hhv. langs x-, y- og z-aksen. De er på \ref{fig:2porbitaler} illustreret i et kartesisk koordinatsystem.
  
\begin{figure}[ht!]
  \centering
  \begin{tikzpicture}[scale=1.5]
  \begin{scope}[font=\itshape]% to not type it every time, but better go for math mode

 \draw[-latex] (0,0)--(1,0) node[above]{x};
  \draw[-latex] (0,0)--(-0.5,-0.7) node[above]{y};
  \draw[-latex] (0,0)--(0,1) node[above]{z};
  \orbital[pcolor = purple, pos = {(0,0)}]{py}
  \node[above] at (0.7,0.7) {p$_x$};
	
  \draw[-latex] (3,0)--(3.5,0.7);
  \draw[-latex] (3,0)--(4,0) node[above]{x};
  \draw[-latex] (3,0)--(2.5,-0.7) node[above]{y};
  \draw[-latex] (3,0)--(3,1) node[above]{z};
  \orbital[pcolor = purple, pos = {(3,0)}]{px} 
  \node[above] at (3.7,0.7) {p$_y$};

  \draw[-latex] (6,0)--(7,0) node[above]{x};
  \draw[-latex] (6,0)--(5.5,-0.7) node[above]{y};
  \draw[-latex] (6,0)--(6,1) node[above]{z};
  \orbital[pcolor = purple, pos = {(6,0)}]{pz}
  \node[above] at (6.7,0.7) {p$_z$};
  \end{scope}
  % correctly setting the background layer
  \begin{pgfonlayer}{backbackground}
  \fill[white](current bounding box.south west)rectangle
  (current bounding box.north east);
  \end{pgfonlayer}
  \end{tikzpicture}
  \caption{$2p_x$, $2p_y$, $2p_z$ orbital \label{fig:2porbitaler}} \end{figure}

I 3 skal kan der være 18 elektroner. De første 2 elektroner bliver først fyldt ind i en 3s orbital, der ligner både 1s og 2s orbitalen i og med, at den er kugleformet. Radius på 3s orbitalen er større end på 2s orbitalen grundet det højere energiniveau. 6 af de resterende 16 elektroner kan dernæst findes i 3p orbitalerne. Der er, ligesom 2p-orbitalerne, også tre 3p-orbitaler - $3p_x$,$3p_y$ og en $3p_z$ orbital. For det illustrative formål kan vi bare betragte 2p-orbitalerne. 3p-orbitalerne ligner dem, men elektronerne kan bare befinde sig i større sløjfer end 2p-orbitalerne. 
\\
Nu har vi kigget på de simpleste af orbitalerne, s- og p-orbitalerne. Der eksisterer også d og f orbitaler. f-orbitalerne vil der blive set bort fra i denne opgave, men d-orbitalerne er illustretet i figur \ref{fig:dorbitaler}.
\\
\begin{figure}[ht!]
  \centering
  \begin{tikzpicture}[scale=1.5]
  \begin{scope}[font=\itshape]% to not type it every time, but better go for math mode

 \draw[-latex] (0,0)--(1,0) node[above]{x};
  \draw[-latex] (0,0)--(-0.5,-0.7) node[above]{y};
  \draw[-latex] (0,0)--(0,1) node[above]{z};
  \orbital[pcolor = purple, pos = {(0,0)}]{dxy}
  \node[above] at (0.7,0.7) {$d_{xy}$};
	
  \draw[-latex] (3,0)--(3.5,0.7);
  \draw[-latex] (3,0)--(4,0) node[above]{x};
  \draw[-latex] (3,0)--(2.5,-0.7) node[above]{y};
  \draw[-latex] (3,0)--(3,1) node[above]{z};
  \orbital[pcolor = purple, pos = {(3,0)}]{dxz} 
  \node[above] at (3.7,0.7) {$d_{xz}$};

  \draw[-latex] (6,0)--(7,0) node[above]{x};
  \draw[-latex] (6,0)--(5.5,-0.7) node[above]{y};
  \draw[-latex] (6,0)--(6,1) node[above]{z};
  \orbital[pcolor = purple, pos = {(6,0)}]{dyz}
  \node[above] at (6.7,0.7) {$d_{yz}$};
  
  \draw[-latex] (1,-2)--(2,-2) node[above]{x};
  \draw[-latex] (1,-2)--(0.5,-2.7) node[above]{y};
  \draw[-latex] (1,-2)--(1,-1) node[above]{z};
  \orbital[pcolor = purple, pos = {(1,-2)}]{dx2y2}
  \node[above] at (1.7,-1.3) {$x^{2}-y^{2}$};

  \draw[-latex] (5,-2)--(6,-2) node[above]{x};
  \draw[-latex] (5,-2)--(4.5,-2.7) node[above]{y};
  \draw[-latex] (5,-2)--(5,-1) node[above]{z};
  \orbital[pcolor = purple, pos = {(5,-2)}]{dz2}
  \node[above] at (5.7,-1.3) {$z^2$};
  
  \end{scope}
  % correctly setting the background layer
  \begin{pgfonlayer}{backbackground}
  \fill[white](current bounding box.south west)rectangle
  (current bounding box.north east);
  \end{pgfonlayer}
  \end{tikzpicture}
  \caption{$d_{xy}$, $d_{xz}$, $d_{yz}$, $x^{2}-y^{2}$, $z^2$ orbitalerne \label{fig:dorbitaler}} 
  \end{figure}