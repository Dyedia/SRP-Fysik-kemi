\section{Den Organiske kemi}
Organisk kemi er $carbonforbindelsernes$ $kemi$(Quote kemi A bog). Når vi ser på organisk kemi vil alle de bindinger vi betragter være kovalente. Vi 


\subsection{Kovalente bindinger}
Så det vi fandt ud af i dette afsnit var, at når 2 atomer går sammen og deler deres elektroner for at opnå en lavere energitilstand dannes nogle bindinger som vi kalder kovalente bindinger. Kovalente bindinger dannes når atomer, der har lige stor tilbøjelighed til at afgive elektroner mødes. Et eksempel på et organisk molekyle, der dannes kovalente bindinger er ethylen. Her ses det tydeligt hvordan de to p-orbitaler går sammen i midten af molekylet om at lave en $\sigma_p$-(se figur 6). Delte elektroner findes primært mellem de positivt ladede atomkerner og den elektrostatiske tiltrækning mellem de to positivt ladede kerner og de to negativt ladede elektroner holder molekylet sammen. Det bånd der bliver dannet som et resultat af førnævnte er meget stærkt.
