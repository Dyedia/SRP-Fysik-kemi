\section{Eksempler fra den organiske kemi}
Den organiske kemi er $carbonforbindelsernes$ $kemi$(Quote kemi A bog). Lad os indledningsvist se på methan, der har molekylformlen $CH_4$. Carbon skal kunne danne forbindelse til alle 4 hydrogenatomer, men hvis vi betragter carbons elektronkonfiguration vil vi lægge mærke til, at der kun er 2 ledige elektroner til at danne bindinger. Dette kan vi se ved at skrive de orbitaler op, som carbon fylder ud.
\\

Carbon har 6 elektroner og vil derfor først fylde 1s-orbitalen ud, dernæst 2s-orbitalen og til sidst vil der være 1 elektron i 2 af 2p-orbitalerne. Det er de 2 elektroner, der er alene i 2-orbitalerne der har mulighed for at danne en binding. Vi var dog interesserede i at danne 4 bindinger - netop til de 4 hydrogenatomer. Derfor exciteres ét af de elektroner, som befinder sig i 2s-orbitalen til en 2p-orbital sådan at der er 4 orbitaler med 1 elektron i hver. Denne excitering danner grundlag for en ny hybridorbital. Dette er $sp^3$-orbitalen. Navnet henviser til forholdet mellem s- og p-orbitaler. Der er 4 af disse. Hvis betragter carbons elektronkonfiguration nu vil vi se, at der først bliver fyldt op i $1s$-orbitalen og dernæst fyldes 1 elektron i hver af de fire $sp^3$-orbitaler. Dette giver anledning til at carbon nu kan danne 4 bindinger. Hver af de 4 $sp^3$-orbitaler går nu sammen med en $1s$-orbital som hydrogenatomerne har og danner en $\sigma$-binding således at molekylet stabiliseres. For at molekylet får en så lav energi som muligt vil bindingerne placere sig så langt væk fra hinanden som muligt og dette resulterer i at vi får den tetraidstruktur methan er kendt for at have (FODNOTE). 
\\

Et andet eksempel er ethen, hvis molekyl struktur er $C_{2}H_2$. Her har vi 2 carbonatomer der er bundet sammen af en dobbeltbinding. I dette eksempel skal hver carbonatom danne 4 bindinger - ét til hvert af de hydrogenatomer, der er bundet til carbonatomet og 2 til en $\sigma$- og en $\pi$-binding melem de 2 carbonatomer. Betragtes carbons elektronkonfiguration kan vi ligesom i det forrige eksempel se, at der kun er mulighed for at danne 2 bindinger. Derfor exciteres en elektron fra $2s$-orbitalen til en $2p$-orbital. Men da carbonatomet bliver nød til at behold en $2p$-orbital for at kunne danne en $\pi$-binding beholder den én af de 3 $2p-orbitaler$ der nu eksisterer. De 2 resterende $2p$-orbitaler går sammen med $2s$-orbitalen om at danne 3 $sp^{2}$-orbitaler. Således går 2 af de 3 $sp^2$-orbitaler til at danne $\sigma$-bindinger med de 2 hydrogenatomer og den sidste $sp^2$-orbital går sammen med det andet carbonatoms $sp^2$-orbital om at danne en $\sigma$-binding. Den $2p$-orbital, som ikke hybridiserede danner nu en $\pi$-binding der hvor der overlappes med det andet carbonatoms $2p$-orbital. 


\subsection{Kovalente bindinger}
Så det vi fandt ud af i dette afsnit var, at når 2 atomer går sammen og deler deres elektroner for at opnå en lavere energitilstand dannes nogle bindinger som vi kalder kovalente bindinger. Kovalente bindinger dannes når atomer, der har lige stor tilbøjelighed til at afgive elektroner mødes. Et eksempel på et organisk molekyle, der dannes kovalente bindinger er ethen. Her ses det tydeligt hvordan de to p-orbitaler går sammen i midten af molekylet om at lave en $\sigma_p$-(se figur 6). Vi fandt også ud af, i eksemplet med methan og ethen, at elektroner kan exciteres fra deres orbitaler for at danne nye hybridiserede orbitaler, som gør molekylet mere stabilt. Denne process kaldte vi hybridisering. Vi fandt også ud af at delte elektroner primært befandt sig mellem de positivt ladede atomkerner og den elektrostatiske tiltrækning mellem de to positivt ladede kerner og de to negativt ladede elektroner holder molekylet sammen samt at det bånd der blev dannet som et resultat af førnævnte er meget stærkt.